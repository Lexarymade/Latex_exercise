\documentclass{article}
%\usepackage[space]{ctex}
\usepackage{amsmath}
\usepackage{multirow}
\usepackage{booktabs}
\usepackage{graphicx}
\usepackage{fontspec} %设置字体要导入这个包   
\usepackage{makecell} %这个包里的Xhline可以改变表格横线的粗细
\usepackage[normalem]{ulem}
\useunder{\uline}{\ul}{}

\title{\vspace{-4cm}Lexarymade's reflection 2023}
\date{}
\author{Lexarymade, Bao Zhongyuan}

\setmainfont{Times New Roman} %设置字体为tnm



\begin{document}
    \maketitle

       \begin{abstract}
        In this paper, we introduce a brand new algorithm
        called Lexary for image segmentation. The first part
        introduces the background of our research.
    \end{abstract}

    \section{Introduction}
        \subsection{Background}
            \hspace{1.5em}hello, \par world;
            \begin{equation}
                a + b = b  + a 
            \end{equation}

    \section{Table Practice}
    \hspace{1.5em}This is an example of creating table in latex.
    This is an example of creating table in latex.This is an example of creating table in latex.
    This is an example of creating table in latex.

    This is an example of creating table in latex.
    \begin{table}[h!]
        \renewcommand{\arraystretch}{1.3}
        \vspace{-1.5em}
        \centering
        \caption{table practice}
        \vspace{0.5mm}
        \begin{tabular}{cccc}
        \hline
        \textbf{Name} & \textbf{Category} & \textbf{Name} & \textbf{Category} \\ \hline
        Name1         & Category1         & Name4         & Category3         \\
        Name2         & Category2         & Name5         & Category4         \\
        Name3         & Category1         &               &                   \\ \hline
        \end{tabular}
        \vspace{-0.5em}
    \end{table}
    \par This is an example of creating table in latex.


\section{Table Exercise}
    \begin{table}[h!]
        \renewcommand{\arraystretch}{1.3}
        \vspace{-1.5em}
        \centering
        \caption{Mimic table of KD}
        \vspace{1mm}
        \setlength{\tabcolsep}{2.5mm}
        {
            \begin{tabular}{ccccc}
            \Xhline{1pt}
                        & \multicolumn{2}{c}{Teacher} & \multicolumn{2}{c}{\textbf{Student}} \\ [-0.5ex]
                        & Modality          & mIoU(\%)         & Modality          & mIoU(\%)         \\ \Xhline{0.5pt}
                        \pmb{No-KD}           & -                 & -                & RGB               &   47.46               \\
                        No-KD      &   D      & 33.73                 & RGB               &        46.89          \\
                    Modality-general-KD    &       RGB / D            &     34.01             & RGB                  &        47.99($\uparrow $)        \\ 
            \Xhline{1pt}
            \end{tabular}
            \vspace{-0.5em}
        }
    \end{table}


    
 
\end{document}